\section{The economics of migration} \label{eco_migration}

As far as we could observe, there is no clear and formal definition in the literature of the research field of the \textit{economics of migration}, and we do not intend to provide one. However, considering the purpose of this study, we must establish the boundaries of the field with which we are working. In our understanding, we can talk about the \textit{economics of migration} as a research field since the 1960s, when the first proper economic models were developed, and scientific research began to gain momentum. Formally speaking, \textit{we consider the economics of migration to be a research field to which are associated scientific works that involve at least one economic aspect of migration, or migratory phenomena, namely: economic determinants of migration, economic consequences of migration, economic theories and models of migration, and economic policies on migration}\footnote{Although we did not address policy-related subjects throughout the literature review, it was clear that this topic has always been part of the debate, particularly in studies on migration in developing countries. Hence, it is only fair to consider it as part of our definition.}. 

With that in mind, it is safe to say that, for the 20th century, the \textit{economics of migration} was internal migration, and not just any type of migration, but labor migration\footnote{As noted by Lucas in his survey on internal migration in developing countries, ``indeed in general terms it is probably fair to say that economists have been largely preoccupied with the migration of labor" \citep[p. 786]{lucas_internal_1997}.}. Even though the economic research on migration was biased towards internal migration during the 20th century, there were indications that a shift was taking place \citep{cushing_crossing_2004}. In fact, we were able to observe from our review that the most significant advances in international migration research began to occur at the end of the 20th century, specifically from the mid-1980s onwards. Therefore, it is natural to hypothesize that the increasing interest in international migration observed at the end of the 20th century has continued throughout the 21st century, meaning that the \textit{economics of migration} has devoted more attention to international migration.  