\section{The economics of migration} \label{eco_migration}

Economics is one of many disciplines interested in migration, and sometimes the multidisciplinary nature of the field of migration studies makes it difficult to recognize specific disciplinary contributions. As far as we could observe, there is no clear and formal definition in the literature of the research field of the \textit{economics of migration}, and we do not intend to provide one. However, considering the purpose of this study, it is necessary that we at least establish the boundaries of the field with which we are working. Therefore, considering what we have seen in the literature, in this study we understand scientific works belonging to the economics of migration to be any work, whether theoretical or empirical, regardless of the classification (for example, internal or international migration), that involves at least one economic aspect of migration, among which we highlight in particular: economic determinants of migration, economic consequences of migration, economic theories and models of migration, and economic policies on migration\footnote{Although we did not address policy-related subjects throughout the literature review, it was clear that this topic has always been part of the debate, particularly in studies on migration in developing countries, and therefore it is only fair to consider it as part of our definition.}.

Considering what we presented in Section \ref{lit_review}, it is safe to say that, for the economic research on migration in the 20th century, migration was internal migration, and not just any type of migration, but labor migration\footnote{As noted by Lucas in his survey on internal migration in developing countries, ``indeed in general terms it is probably fair to say that economists have been largely preoccupied with the migration of labor" \citep[p. 786]{lucas_internal_1997}.}. Overall, we can conclude that the literature on internal migration during the 20th century developed more extensively and fruitfully than the one on international migration, both from a theoretical and empirical point of view. Concerning the theoretical literature, we showed that the most widely used and applied theoretical frameworks and models were developed for internal migration and were later adapted for use in international migration. Regarding the empirical literature, the evidence from internal migration seems more robust, which can be due to the fact that the interest for international migration appears to have come later; for example, until the publication of their survey in 1997, LaLonde and Topel had not found any robust evidence regarding the determinants of international migration, a clear contrast with the literature on internal migration, which indicates that the empirical research on international migration was lagging behind \citep{lalonde_economic_1997}.

Even though the economic research on migration was biased towards internal migration during the 20th century, there were indications that a shift could have been taking place. As pointed out by Cushing and Poot, ``recent years have seen a marked shift in migration research from population redistribution within national borders (internal migration) to movement across borders (international migration)" \citep[p. 318]{cushing_crossing_2004}. In fact, we were able to see from our review that the greatest advances in international migration research began to occur at the end of the 20th century, specifically from the mid-1980s onwards. Therefore, it is natural to hypothesize that the increasing interest in international migration observed at the end of the 20th century has continued throughout the 21st century, meaning that the \textit{economics of migration} has devoted more attention to international migration.