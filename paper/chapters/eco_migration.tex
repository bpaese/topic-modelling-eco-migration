\section{The economics of migration} \label{eco_migration}

Economics is one of many disciplines interested in migration, and sometimes the multidisciplinary nature of the field of migration studies makes it difficult to recognize specific disciplinary contributions. As far as we could observe, there is no clear and formal definition in the literature of the research field of the \textit{economics of migration}, and we do not intend to provide one. However, considering the purpose of this study, it is necessary that we at least establish the boundaries of the field with which we are working. Therefore, considering what we have seen in the literature, in this study we understand scientific works belonging to the economics of migration to be any work, whether theoretical or empirical, that involves at least one economic aspect of migration, namely: (i) economic determinants of migration; (ii) economic consequences of migration; (iii) economic policies on migration; and (iv) economic theories and models in migration.

Considering what we presented in Section \ref{lit_review}, it is safe to say that, for economics in the 20th century, migration was internal migration, and not just any type of migration, but labor migration\footnote{As noted by Lucas, ``indeed in general terms it is probably fair to say that economists have been largely preoccupied with the migration of labor" \citep[p. 786]{lucas_internal_1997}.}. Economic research on migration has devoted much more effort to studies on internal migration, both theoretical and empirical. For example, from the review of the theoretical literature, we showed that the most widely used and applied theoretical frameworks and models, which have even broken down disciplinary barriers, were developed for internal migration and were later adapted for use in international migration. Regarding the empirical literature, we also noticed that the literature on internal migration developed more extensively and fruitfully, since interest in international migration seems to have come later. For instance, until the publication of their survey in 1997, LaLonde and Topel had not found any robust evidence regarding the determinants of international migration, a clear contrast with the literature on internal migration, which is an indicative that the empirical research on international migration was lagging behind \citep{lalonde_economic_1997}.

As pointed out by Cushing and Poot, ``recent years have seen a marked shift in migration research from population redistribution within national borders (internal migration) to movement across borders (international migration)" \citep[p. 318]{cushing_crossing_2004}. In fact, we were able to see from the literature review that the greatest advances in international migration research began to occur at the end of the 20th century, specifically from the mid-1980s onwards. Therefore, it is natural to hypothesize that the increasing interest in international migration observed at the end of the 20th century has continued throughout the 21st century, and thus that the \textit{economics of migration} has devoted more attention to international migration issues, as highlighted by \cite{cushing_crossing_2004}.