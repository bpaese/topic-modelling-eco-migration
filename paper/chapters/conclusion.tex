\section{Conclusion} \label{conclusion}

During the 20th century, the economic research on migration was dominated by internal migration, both from a theoretical and empirical perspective. However, by the end of the century, the literature on international migration began to flourish, gaining more space on the research agenda. In order to understand how the scientific research in the field of \textit{economics of migration} has evolved since the end of the 20th century, we estimated an LDA topic model, through which we were able to uncover the field's topical composition By identifying the topics that were more oriented toward internal or international migration, we found that, in fact, throughout the 21st century, topics related to international migration have increased their share, while topics related to internal migration have lost space.

Providing answers as to why the economics of migration has shifted more toward international migration is beyond the scope of this work, but we can shed light on some possible answers. For instance, estimates show a sustained increase in the number of international migrants since the 1970s, jumping from approximately 84 million in 1970 to 173 million in 2000, and reaching 281 million in 2020. This increase in international migration flows may have caught the attention of economists to issues and problems surrounding the phenomenon, such as international remittances (as in topics 18 and 21), the impacts of immigration in the receiving countries (as in topics 6 and 23), or the movement of people who feel forced to move, like refugees and asylum seekers (as in topic 20) \citep{mcauliffe_world_2024}. Another possible explanation is that since the \textit{economics of migration} is an integral part of the field of migration studies, which has become more interdisciplinary \citep{levy_between_2020}, then, since migration studies shifted to international migration \citep{cushing_crossing_2004, king_mind_2010}, the same may have happened with the \textit{economics of migration}, due to the influence of other disciplines and areas of study. In reality, this interdisciplinary characteristic may also be leading to a greater diversification in the \textit{economics of migration}, given the increase seen in the likelihood of studies being associated with topics traditionally linked to other social sciences, such as topics 10, 12, 14, and 20.

Regarding the limitations of this study, we acknowledge that the Web of Science database is biased toward American and European journals, which are English-language based, possibly under-representing Global South scientific research. Another limitation come from the interpretation and labelling of topics, which depend heavily on the authors' prior knowledge of the literature. Future research may focus on the estimation of other topic modelling techniques, such as Dynamic Topic Modelling (DTM), which is likely to work better to analyze the evolution of topics over time. Additionally, future works may build upon ours to better explain the changes we observed.