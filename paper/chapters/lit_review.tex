\section{Literature review} \label{lit_review}

Even though migration emerged as a field of study in the late 19th, early 20th centuries, the interest of economists for migration began to take form only in the 1930s. The Depression of the 1930s caused unemployment rates to reach notable high levels in important industrialized countries, which fuelled rural-urban migration. On the other hand, these movements of unemployed people from rural areas toward cities, in search of job opportunities and better living conditions, contributed to an increase in the already high unemployment levels \citep{greenwood_early_2003}. This scenario raised numerous policy concerns among economists, fostering their interest in migration research \citep{lucas_internal_1997, greenwood_internal_1997, greenwood_early_2003}. However, despite the increasing interest in migration in the first half of the 20th century, actual research faced many obstacles, such as the lack of proper theoretical frameworks \citep{greenwood_early_2003}.

Considering that we are proposing an investigation on how the scientific research on the \textit{economics of migration} evolved, it is importance that we understand how the literature developed. Hence, we review the main contributions to the economic research on migration in this section, from the early days of migration studies until approximately the end of the 20th century.  Given that in economics the literature is traditionally split into internal and international migration, we decided to follow this dichotomy because we understand that through it we can better comprehend the evolution of scientific research in the field of \textit{economics of migration}. For organizational reasons, we have decided to present the review of the theoretical literature at first, and then we move on to the review of the empirical literature

\subsection{Theoretical literature} \label{lit_review_theories}

\subsubsection{The influence of mainstream economic theory and Sjaastad's human capital model}

In 1932, Hicks wrote that ``recent researches are indicating more and more clearly that differences in net economic advantages, chiefly differences in wages, are the main causes of migration" \citep[p. 76]{hicks_theory_1932}. According to \cite{greenwood_research_1975}, this concisely states the position of mainstream (or orthodox) economic theory on migration\footnote{Actually, \cite{greenwood_research_1975} used the term ``geographical mobility of labor" instead of ``migration", because in the neoclassical framework, every individual is seen as a worker, who is a potential migrant, and so migration is treated a movement of workers, which from a macro perspective means a supply of labor. That's the reason why sometimes the terms ``migration", ``labor migration", and ``labor mobility" are used interchangeably in the economic literature on migration.} until at least the 1970s, when most, if not all, migration research done in economics adopted the neoclassical framework\footnote{The neoclassical framework, which dominated mainstream economic theory for most of the 20th century, is based on assumptions of individual utility maximization and rational choice, according to which rational individuals seek to maximize their (expected) utility. From the point of view of an individual, or worker, maximizing utility meant maximizing income, and so an utility-maximizing behaviour would be seeking a higher income.} to overcome the lack of a suitable theory of migration. Under this theoretical framework, also called \textit{disequilibrium system} \citep{greenwood_internal_1997}, migration is driven by interregional wage differentials, which, from the perspective of rational individuals that maximize utility, represent opportunities for utility gains \citep{greenwood_research_1975, lucas_internal_1997, greenwood_internal_1997}.

It wasn't until the 1960s that important theoretical advances made it possible to talk about an \textit{economics of migration}. \cite{sjaastad_costs_1962} came up with what is arguably the first proper economic model in migration by applying the notion of investment in human capital to the decision to migrate \citep{greenwood_internal_1997}. The author proposed to identify important costs and returns (or benefits) to migration, reducing the decision to migrate to a simple cost-return analysis. In this setting, wage differentials over space account for private money returns to migration, that is, the possibility of obtaining higher earnings in a different location is seen as a return to migration, thus impacting the decision to migrate\footnote{Greenwood notes that the influence of the neoclassical tradition is evident in Sjaastad's model, since ``economic opportunity differentials represent potential for household utility gains that can be arbitraged by migration" \citep[p. 670]{greenwood_internal_1997}.}. Therefore, in Sjaastad's model, migration helps individuals seek the highest returns on their human capital \citep{sjaastad_costs_1962}. For this reason, some authors referred to it as the \textit{human capital model} of migration \citep{shields_emergence_1989, greenwood_internal_1997}.

Sjaastad's prominent work was so influential for the economic research on migration that Greenwood wrote that after its publication ``migration research by economists really began to blossom" \citep[p. 669]{greenwood_internal_1997}. Although it was not a theoretical framework with a sophisticated mathematical foundation, it provided a solid conceptual basis on which other economists could build, and did build.

\subsubsection{Internal migration in developing countries: the Harris-Todaro model} \label{lit_review_theories_HT}

Until the mid-1960s, most efforts had been directed toward studying migration in developed, industrialized countries, since it was the spread of urbanization in these countries that had stimulated research on migration, as previously stated. However, during the 1960s, many so-called developing countries began to undergo the same processes of urbanization and industrialization, drawing the attention of scholars to the relationship between internal migration, especially the rural-urban migration characteristic of urbanization processes, economic growth, and economic development \citep{lewis_economic_1954, ranis_theory_1961, easterlin_internal_1980}. Therefore, in the economic literature on migration, migration and economic development walked side by side from the second half of the 1950s until at least the mid-1970s, motivated by an interest in the role of labor supply in the process of economic development and growth in developing countries \citep{lewis_economic_1954, ranis_theory_1961}. In this context, internal migration was seen as a beneficial, even desirable process, according to which cheap labor from agricultural sectors would flow to industrial, urban regions, fuelling industrialization, and consequently development and growth \citep{easterlin_internal_1980}. During this period, notable models based on the foundations established by \cite{sjaastad_costs_1962} were proposed \citep{lucas_internal_1997}.

In the late 1960s, \cite{todaro_model_1969} developed the \textit{Todaro migration model}, which, according to \cite{stark_migration_1991}, kicked off the movement of intensive research in the field of rural-urban migration in developing countries. The model consists of a two-sector framework--one industrialized urban sector and one underdeveloped rural sector--and it is fundamentally a behavioral model of labor supply. Among the main features is that migration is primarily stimulated by rational economic considerations of relative costs and benefits \citep{easterlin_internal_1980}, showing off the influence of Sjaastad's human capital model \citep{shields_emergence_1989, lucas_internal_1997}. \cite{todaro_model_1969} introduces the novel idea that the decision to migrate is conditioned by the urban-rural differences in \textit{expected} income\footnote{For this reason, \cite{shields_emergence_1989} called it the \textit{expected income model}.}, which are made possible by what he defined as \textit{expected} urban earnings. According to Todaro, ``the fundamental premise is that as decision-makers migrants consider the various labor-market opportunities available to them as, say, between the rural and urban sectors, choosing the one that maximizes their ``expected" gains from migration" \citep[p. 364]{easterlin_internal_1980}, meaning that potential rural migrants are maximizers of \textit{expected} utility \citep{easterlin_internal_1980}. Therefore, as long as there are regional differences in \textit{expected} earnings, that is, the expected earnings in the urban sector exceed the actual earnings in the agricultural sector, there will be rural-urban migration, despite the levels of urban unemployment\footnote{As we can see, an important difference between Todaro's model and Sjaastad's model is that, for Sjaastad, migrants take into consideration the \textit{actual} rural-urban wage differentials in their migration decision \citep{lucas_internal_1997}.}.

Following \cite{todaro_model_1969}, \cite{harris_migration_1970} presented their seminal \textit{Harris-Todaro model} of migration, which is an extended version of the basic Todaro model where potential rural migrants continue to behave as maximizers of expected utility. However, \cite{harris_migration_1970} present a comprehensive two-sector general equilibrium model, integrating both sectors into a cohesive analytical that investigates the migration impact on production and welfare \citep{harris_migration_1970, easterlin_internal_1980}. The Harris-Todaro model of migration was as influential as Sjaastad's human capital model for the economic research on migration. It has been ever since the standard neoclassical two-sector model of migration, contributing with important public policy recommendations for developing countries \citep{harris_migration_1970, easterlin_internal_1980}. Furthermore, like previous models, this one has also been subject to modifications and extensions, such as that made by \cite{corden_urban_1975}, who introduced capital mobility between the rural and urban sectors, and the one proposed by \cite{fields_rural-urban_1975}, which was praised by \cite{easterlin_internal_1980} as the most extensive and useful modification of Harris-Todaro model.

\subsubsection{The New Economics of Labor Migration} \label{lit_review_theories_NELM}

As we have shown, theoretical frameworks based on neoclassical economic theory simply dominated migration research during the 20th century, and not only in economics. However, this class of theories and models also had their critics. For instance, \cite{stark_new_1985} and \cite{stark_migration_1991} questioned the assumptions and conclusions of neoclassical theories, arguing that the axiomatic of neoclassical models caused them to overlook empirical evidence, leading to conflicting or fallacious views between rural-urban migration and topics such as fertility, education, and urban employment, among others.

These critiques were part of a broader movement that proposed a new theoretical framework for migration. This new theoretical framework came to be known as \textit{New Economics of Labor Migration} (NELM), or just \textit{New Economics of Migration}, which is a paradigm shift in theories of migration \citep{stark_new_1985, stark_migration_1991}. The most significant change introduced by this new framework concerns the unit responsible for the decision to migrate, which under the neoclassical tradition is a single individual, that happens to be a rational utility-maximizer. According to Stark, ``migration decisions are often made jointly by the migrant and by some group of non-migrants. Costs and returns are shared, with the rule governing the distribution of both spelled out in an implicit contractual arrangement between the two parties" \citep[p. 25]{stark_migration_1991}. Therefore, proponents of the NELM highlight that migration decisions are not individual decisions, but joint decisions instead, taken within the ambit of the household or family \citep{stark_new_1985, stark_migration_1988, stark_migration_1991}. 

Just like other models previously mentioned, the NELM primarily addressed questions and issues regarding rural-urban migration\footnote{It is worth noting that the theoretical framework of NELM would also end up being adapted for the analysis of international migration, as shown by \cite{massey_theories_1993} and \cite{king_theories_2012}.} in least developed countries (LDCs). Economists noted that migration played a key role in mitigating risks in this group of countries. Therefore, contrary to what neoclassical tradition said, the incentive for rural-urban migration was not so much the interregional wage differentials, but rather the possibility of diversifying the family's income-generating activities, leading to the minimization of the risks to which the household was exposed, such as crop failures. In other words, migration--in this case, sending a family member to work in the urban center--would provide the household with a reliable source of liquidity, in the form of \textit{remittances}, which could be used to offer income insurance, or to finance new production technologies, inputs and activities \citep{stark_migration_1982, stark_new_1985, stark_migration_1988, stark_migration_1991, taylor_new_1999}.

\subsubsection{Theoretical frameworks on international migration} \label{lit_review_theories_international}

All of the aforementioned models were originally conceived to be applied to cases of internal migration, given the factors already presented that motivated the study of migration in economics. Even though the work considered by many to be the first on international migration was published in 1927 (see Appendix \ref{migration_studies}), the theoretical literature on international migration did not developed as prolifically as that on internal migration. As a matter of fact, the most basic model for studying international migration was built borrowing from models of internal migration\footnote{Apparently, this was not unique to the economics of migration. As shown by \cite{massey_theories_1993} and \cite{king_theories_2012}, other social sciences also adopted and adapted theoretical frameworks and models initially developed by economists for the study of internal migration to international migration. In fact, it seems that economic theories and models have been prominent in migration studies \citep{massey_theories_1993, king_theories_2012}.}, especially Sjaastad's human capital model and the Todaro migration model. According to this basic framework, international migration would be driven by differences in the expected earnings between sending and receiving countries \citep{lalonde_economic_1997}.

The main contributions to the theoretical literature ended up being developed with a focus on analyzing the economic impacts of immigrants on the native populations, or the receiving country as a whole. As pointed out by \cite{card_is_2005}, one of the main approaches to estimate the impact of immigration on native workers is the so-called \textit{local labor markets approach}, which was pioneered by \cite{grossman_substitutability_1982}. As stated by Card, this approach ``relates differences in the relative structure of wages in different local labor markets to differences in the relative supply of immigrants" \citep[p. F302]{card_is_2005}, which is closely related to other works on internal migration and wage structures, including \cite{sjaastad_costs_1962}. Among notable studies that adopted this local labor markets approach is the work of \cite{borjas_self-selection_1987}, which developed his \textit{self-selection} model, a model of selection of immigrants on the basis of unobserved characteristics\footnote{\cite{borjas_self-selection_1987} was careful to note that his model adopts the cornerstone hypothesis of income maximization of Sjaastad's human capital model, once again making clear the influence of internal migration models on theoretical frameworks in international migration.}--such as underlined abilities and productivities--built on the classic work of \citep{roy_thoughts_1951}. Later on, \cite{borjas_immigration_1991} expanded the self-selection model to account for observed characteristics, such as education \citep{borjas_economics_1994}.

\subsection{Empirical literature on internal migration} \label{lit_review_empirical_internal}

\subsubsection{Early contributions and modified gravity models}

From the early days of migration studies until the end of the 1950s, nearly all research efforts on migration was directed towards a better understanding of the causes, or determinants, of migration \citep{sjaastad_costs_1962, greenwood_research_1975, greenwood_early_2003}. Besides, although some studies did deal with flows of immigrants (see Appendix \ref{migration_studies}), internal migration dominated the agenda, which is consistent with the motivation behind migration research, as previously discussed. However, the lack of theories and appropriate data limited the scope of studies, which were mainly descriptive in style and rather informal in technique. Among the possible determinants considered in early contributions were distance and population size, which are well-known gravity variables, and a couple of personal demographic characteristics, such as age, education, race, income, and marital status \citep{greenwood_internal_1997}.

The fact that two of the determinants mentioned above were gravity variables was not a mere coincidence. Gravity models are considered by some to be the first formal class of models applied to migration \citep{greenwood_early_2003, cushing_crossing_2004}. These models were limited to analysis from a macro perspective, focusing on the role of space in driving migration, and relied on aggregate data, which was more accessible at the time \citep{greenwood_research_1975, greenwood_internal_1997, greenwood_early_2003, cushing_crossing_2004}. According to \cite{greenwood_early_2003}, the intuition of gravity models of migration can be traced back to Ravenstein's \textit{laws of migration} \citep{ravenstein_laws_1885, ravenstein_laws_1889}. However, the adaptation required to their use in the social sciences, and particularly in migration studies, only occurred in the 1940s, when \cite{zipf_unity_1942, zipf_p1_1946} and \cite{stewart_empirical_1947} ``elaborated the model and applied it to migration and other human spatial interactions" \citep[p. 26]{greenwood_early_2003}.

The emergence of the first theoretical frameworks during the 1960s motivated the ``modification" of gravity models, with the addition of variables expected to influence the decision to migrate, and the redefinition of other variables, which were given behavioral content. This new class of gravity models, called \textit{modified gravity models}, have played a remarkably important role in migration research at least until around the mid-1970s, in a period when virtually all empirical studies were specified using them \citep{greenwood_research_1975, greenwood_internal_1997, greenwood_early_2003}. As an example, Sjaastad's human capital model ``provided an appealing rationale for the presence of income variables in modified gravity models, as well as in other models of migration" \citep[p. 670]{greenwood_internal_1997}. 

In a context in which modified gravity models prevailed, \cite{greenwood_research_1975} provided a comprehensive survey of economists' main empirical contributions to the literature on internal migration in the United States from 1960 onwards, focusing on the determinants and consequences of migration. Concerning the determinants, Greenwood states that ``one of the clearest implications of the related literature is that gross migration declines perceptibly with increased distance" \citep[p. 410]{greenwood_research_1975}. Additionally, there was sufficient evidence in the literature that personal characteristics, such as income, employment status, age, level of education, and race, were important in determining migration. In what concerns the consequences of migration, \cite{greenwood_research_1975} noted the lack of studies focusing on them, as was already the case before 1960, as pointed out by \cite{sjaastad_costs_1962}. According to Greenwood, it was likely that economists gave more attention to the determinants than to the consequences of migration because investigating the consequences would require overly complex models and types of data that were not widely available at the time \citep{greenwood_research_1975}.

\subsubsection{Improvements in data quality: the role of personal characteristics and life-cycle and family factors}

The lack of adequate data was one of the biggest, if not the biggest, limiting factor to the development of economic research on (internal) migration at the time, since virtually all applied migration research was necessarily based on aggregate data, as stated before. As Greenwood himself would write more than two decades later, the aggregate data on which modified gravity models were based on embodied ``a number of shortcomings that prevented the study of many important issues bearing on migration" \citep[p. 707]{greenwood_internal_1997}. For these reasons, the improvement in the quality of migration data is arguably the most critical change in migration research in the last two decades of the 20th century. 

Particularly since the 1970s, the increasing availability and use of other types of data, especially microdata and longitudinal data, but also time series data, allowed for refinements in econometric techniques, and consequently a better overall understanding of migration processes. This increased availability of data had a considerable impact in several areas. For instance, it allowed a better understanding of how personal characteristics--such as employment status, earnings, education, accumulated skills and training, job tenure, age, sex, and health--and life-cycle and family factors--among which are marriage, divorce, birth and ageing of children, completion of schooling, military service, and retirement--influence the decision to migrate \citep{greenwood_human_1985, greenwood_internal_1997}. 

One relationship that had long troubled economists was that between employment status, or unemployment, and migration\footnote{As Greenwood had noted on his survey, ``one of the most perplexing problems confronting migration scholars is the lack of significance of local unemployment rates in explaining migration" \citep[p. 411]{greenwood_research_1975}.}. About this, scholars identified three possible ``channels"--as defined by \cite{greenwood_internal_1997}--through which unemployment can impact migration: (i) regional unemployment, that is, a region's unemployment rate relative to other regions; (ii) personal unemployment; and (iii) aggregate, or national, unemployment rates \citep{greenwood_internal_1997}. Among these channels, personal unemployment was the one that benefited most from the increased availability of different types of data. According to \cite{herzog_migration_1993}, the notable works of \cite{navratil_socioeconomic_1977}, possibly the first to use microdata to study the relationship between personal unemployment and migration\footnote{Navratil and Doyle also found evidence indicating that the impact of personal and place characteristics on migration, as measured by previous works using aggregate data, are likely biased. Nevertheless, the authors did find that personal characteristics display patterns consistent with past evidences \citep{navratil_socioeconomic_1977}.}, and \cite{davanzo_does_1978}, found evidence that unemployed individuals are considerably more likely to migrate\footnote{For a thorough survey of the empirical literature regarding the relationship between personal unemployment and migration, see \cite{herzog_migration_1993}.}.

Another area that benefited significantly from the improvement in the quality of migration data, as mentioned earlier, is the greater emphasis given to life-cycle and family factors as possible determinants of migration. For instance, the advances made in this area are illustrated by the prominent works of \cite{sandell_women_1977}, which identifies the importance of a wife's employment in the family's migration decision, and \cite{mincer_family_1978}, whose study demonstrates that family ties represent negative externalities for individuals, which tend to discourage migration. 

Besides allowing for a better understanding of how personal characteristics and life-cycle and family factors influence the decision to migrate, the improvement in data quality also made it possible to better understand the unit responsible for the decision to migrate (individual, family, or household), allowed for a better accountability of the experience of migrants, and paved the way for the study of other types of migration, such as return and repeat migration \citep{greenwood_human_1985, greenwood_internal_1997}. Still, most of the progress mentioned kept focusing on the determinants of migration, meaning that the increasing availability of other types of data seen in the last two decades of the 20th century was not enough to stimulate the literature on the consequences of internal migration in developed countries.

\subsection{Empirical literature on international migration} \label{lit_review_empirical_international}

As noted by LaLonde and Topel, ``immigration to the United States during the 1970's and 1980's was greater than in any decade since the 1920's" \citep[p. 297]{lalonde_immigrants_1991}. The economic problems related to these movements of immigrants toward the US caught the attention of economists, prompting a series of empirical studies, aimed mainly at understanding the impact of these immigrants on the US labor market, as well as the problems related to their assimilation, that is, their ability to adapt to the receiving country\footnote{As pointed out in Appendix \ref{migration_studies}, here we can see another example of how an academically dominant country, like the US, has the power to dictate an entire research agenda.} \citep{greenwood_factor_1986, lalonde_economic_1997}.

\subsubsection{Labor market adjustments}

What are the effects of immigration in wage rates and employment in the receiving country? The debate surrounding this question is among the oldest ones concerning international migration, largely due to its policy concerns \citep{greenwood_factor_1986}. From a classic perspective, immigration flows represent an adjustment in the labor markets of both the origin and receiving countries. Particularly with respect to the receiving country, it represents an increase in the labor supply, which can potentially affect wages and displace native workers from jobs, in addition to having significant distributional effects \citep{greenwood_factor_1986, lalonde_immigrants_1991, lalonde_labor_1991, lalonde_economic_1997}. 

Perhaps the first attempt to summarize the empirical literature on the economic impacts of immigration on the labor market of a receiving country (the US, in this case) is the work of \cite{greenwood_factor_1986}. In this opportunity the authors had not found any robust evidence, stating that ``empirical conclusions regarding the effects of immigration on US workers have frequently been based on circumstantial rather than on direct evidence" \citep[p. 1767]{greenwood_factor_1986}. \cite{lalonde_labor_1991} evaluated the effects of immigration to the US by estimating relative wage adjustments among five immigrant cohorts, as well as among young black and Hispanic natives, and concluded that the effects on wages and employment were modest. When surveying the literature, \cite{lalonde_economic_1997} showed that most studies found only small effects of immigration on labor market outcomes of natives.

\subsubsection{Assimilation of immigrants}

A classic definition of assimilation of immigrants that was widely used by economists, however simplistic, is based on their relative
earning power. There are country-specific skills, such as language, institutional knowledge, job-related skills in particular occupations, and culture, which new immigrants naturally lack compared to similar natives. As stated by LaLonde and Topel, ``this means that new immigrants arrive with a human capital `deficit' that reduces their earning power relative to ethnically similar natives" \citep[p. 828]{lalonde_economic_1997}. Thus, the process of assimilation of immigrants involves narrowing the skills gap  between immigrants and native workers, which would lead to a convergence in earnings. One of the most influential studies of immigrant assimilation is \cite{chiswick_effect_1978}, which found that immigrants' relative wages rose with time spent in the US. Chiswick's study also suggests that immigrants acquire a significant amount of country-specific skills during their first decade in the receiving country \citep{chiswick_effect_1978, lalonde_economic_1997}.