\section{Introduction} \label{introduction}

Migration is most certainly not something new. Migratory movements, whether within or across borders, have helped shape the world as we know it today, adding diversity and complexity to societies. In terms of scientific research, economics was among the first disciplines to be interested in migration, therefore being one of the cornerstones of migration studies\footnote{For an overview of migration studies, see appendix \ref{migration_studies}.}\citep{greenwood_early_2003, levy_between_2020, scholten_introduction_2022}.

For reasons unknown to us, economists traditionally split the literature on migration between internal and international migration \citep{cohen_introduction_1996, cushing_crossing_2004, king_mind_2010}. Throughout the 20th century, economic research, whether theoretical or empirical, devoted considerably more attention to internal migration. However, from the mid-1980s onwards, economists began to focus more on issues related to international migration \citep{lalonde_economic_1997, cushing_crossing_2004, card_is_2005}. Therefore, the purpose of this study is to investigate how the scientific research on the \textit{economics of migration} has evolved from the late 20th century to the present day, providing evidence of how the research agenda has behaved regarding the dichotomy between internal and international migration.

Our central hypothesis is that topics related to internal migration, which dominated during the 20th century, have lost their prominence, making way for topics related to international migration to increase their importance in the literature. We propose estimating a Latent Dirichlet Allocation (LDA) topic model, a Bayesian probabilistic model that allows us to identify the latent topics in the scientific literature. Based on our own assessment and interpretation, this method allows us to identify topics related to internal or international migration, thus verifying their behavior over time. To fit our LDA topic model, we use scientific works from the Web of Science (WoS) database, from 1991 to 2024.

This work is structured as follows: in Section 2, we present our literature review, through which we survey how the economic research on migration developed during the 20th century; in Section 3, we sum up the discussion regarding the \textit{economics of migration}; in Section 4, we present our hypothesis and explain our methodology; Section 5 discusses the data we will be using; Section 6 is devoted to our results; in Section 7, we make our concluding remarks, which include perspectives for future research. In the Appendix, we provide a brief overview of migration studies, formally present LDA, and two measures of diversity.