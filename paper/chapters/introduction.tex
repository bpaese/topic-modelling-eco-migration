\section{Introduction} \label{introduction}

Migration is most certainly not something new. Migratory movements have helped shape the world as we know today, besides adding diversity and complexity to societies. Throughout history, people have had various reasons to migrate. Many had to escape poverty, famine, conflicts, or even religious persecution. Some were moved by curiosity and the will to explore the world. However, behind all these motives stands one common reason, which is the search for better life conditions \citep{king_theories_2012, schrover_migration_2022}. If we want to have an idea of how significant the phenomenon has been, we can turn to a few 21st-century numbers. From 2000 to 2024, the estimated number of international migrants jumped from 150 to 281 million; the estimated proportion of the world population considered migrants increased from 2.8\% to 3.6\%; and the total value of international remittances increased by approximately 550\%, rising from 128 to 831 billion US dollars \citep{mcauliffe_world_2024}.

The study of migration has drawn the attention of many areas. Economics was among the first disciplines to be interested in the subject, therefore being one of the cornerstones of migration studies as a research field\footnote{For an overview of the field of migration studies, see Appendix \ref{migration_studies}.}\citep{greenwood_early_2003, levy_between_2020, scholten_introduction_2022}. 