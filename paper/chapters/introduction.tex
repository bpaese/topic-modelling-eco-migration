\section{Introduction} \label{introduction}

Migration is most certainly not something new. Migratory movements, whether within or across borders, have helped shape the world as we know today, adding diversity and complexity to societies \citep{schrover_migration_2022}. The study of migration has always been of interest to many fields. Economics was among the first disciplines to be interested in migration, therefore being one of the cornerstones of migration studies, which emerged as a research field in the late 19th, early 20th centuries\footnote{For an overview of the field of migration studies, see appendix \ref{migration_studies}.}\citep{greenwood_early_2003, levy_between_2020, scholten_introduction_2022}. 

Among the many ways of classifying and organizing the literature on migration studies \citep{cohen_introduction_1996, king_towards_2002, king_mind_2010, king_theories_2012}, the binary between internal and international migration stands out, especially in the \textit{economics of migration}. Throughout the 20th century, economic research on migration, whether theoretical or empirical, devoted considerably more attention to internal migration. However, from the mid-1980s onwards, economists began to focus more on issues related to international migration. Therefore, our main goal in this study is to investigate the evolution of the scientific research on the \textit{economics of migration}, especially throughout the 21st century, in order to provide evidence of a possible shift in the research agenda toward international migration.

Our main hypothesis is that topics related to internal migration, which dominated during the 20th century, have lost their prominence, making way for topics related to international migration to increase their importance in the literature. We propose to estimate a Latent Dirichlet Allocation (LDA) topic model, which is a probabilistic model through which we can investigate how the scientific research on the \textit{economics of migration} has evolved in terms of its topical composition. This method allows us to identify the latent topics in the literature, and, based on our evaluation, sort those related to internal and international migration, thus verifying their behavior over time. To fit our LDA topic model, we use scientific works from 1991 to 2024 from the Web of Science (WoS) database.

This work is structured as follows: in section 2 we present our literature review, through which we survey how the economic research on migration developed during the 20th century; in section 3, we present what we understand by the \textit{economics of migration}, as well as our main hypothesis; in section 4, we explain our methodology; section 5 discusses the data we will be using; section 6 is devoted to our results; in section 7 we make our concluding remarks, which include perspectives for future research. In the appendix, we present an overview of migration studies, a more technical presentation of Latent Dirichlet Allocation, some additional figures.